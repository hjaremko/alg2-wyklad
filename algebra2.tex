\documentclass[a5paper,8pt]{article}
\usepackage[utf8]{inputenc}
\usepackage[T1]{fontenc}
\usepackage[MeX]{polski}
\usepackage{amsmath}
\usepackage{amsthm}
\usepackage{amsfonts}
\usepackage{cases}
\usepackage{geometry}
\newgeometry{tmargin=2cm, bmargin=2cm, lmargin=1.5cm, rmargin=1.5cm}
\setlength{\parindent}{0cm}

\title{Algebra liniowa z geometrią 2}
\date{}
\frenchspacing

\newtheorem{example}{Przykład}[section]
\newtheorem{lemat}{Lemat}[section]
\newtheorem{definition}{Definicja}[section]
\newtheorem{theorem}{Twierdzenie}[section]
\newtheorem*{conclusion}{Wniosek}

\newcommand\tab[1][1cm]{\hspace*{#1}}
\newcommand\defeq{\mathrel{\overset{\makebox[0pt]{\mbox{\normalfont\tiny\sffamily def}}}{=}}}
\newcommand\ass{\mathrel{\overset{\makebox[0pt]{\mbox{\normalfont\tiny\sffamily zał}}}{=}}}
\newcommand\one{\mathrel{\overset{\makebox[0pt]{\mbox{\normalfont\tiny\sffamily 1)}}}{=}}}
\newcommand\two{\mathrel{\overset{\makebox[0pt]{\mbox{\normalfont\tiny\sffamily 2)}}}{=}}}

\begin{document}
    \maketitle
    % \tableofcontents
    % \pagebreak
    % \date{1 marca 2018}

    \section{Terminy kolowium}

    \begin{itemize}
        \item 12.04 - pierwszy semestr
        \item 24.05 - drugi semestr (1-11) Jurlewicz
    \end{itemize}

    \section{Przestrzenie wektorowe} % (fold)
    \label{sec:przestrzenie_wektorowe}

    \large{\textbf{Uwagi:   }}
    $ (V, K, \oplus, \odot) $ - przestrzeń wektorowa

    \begin{enumerate}
        \item Elementy ciala $K$ nazywamy \textbf{skalarami}, a elementy zbioru $V$ \textbf{wektorami}.
        \item Dla uproszczenia zapisu na skalarach i wektorach działania będziemy oznaczali $ +, \cdot $.
        \item Gdy $ K = \mathbb{R} $, to przestrzeń wektorowa $ (V, \mathbb{R}, +, \cdot) $ nazywamy rzeczywistą.
        \item Gdy $ K = \mathbb{C} $, to $ (V, \mathbb{C}, +, \cdot) $ nazywamy zespoloną.
        \item Zamiast pisać $ (V, K, +, \cdot) $ często piszemy "$V$ jest przestrzenią wektorową".
        \item Jeśli $ x, y \in V $, to zapis $x-y$ oznaczamy $x + (-y)$, gdzie $-y$ jest elementem przeciwnym w grupie addytywnej $(V, +)$.
        \item Element neutralny w grupie $(V, +)$ oznaczamy $\Theta$ i nazywamy wektorem zerowym.
    \end{enumerate}

    \begin{example}
        \hfill \break
        $ (\mathbb{R}^N = \underbrace{\mathbb{R} \times \ldots \times \mathbb{R}}_{N}, \mathbb{R}, \oplus, \odot) $ jest przestrzenią wektorową, gdzie działania są zdefiniowane następująco:

        \begin{equation*}
            \begin{split}
                \forall  x ={} & ( x_1, \ldots, x_N), y = (y_1, \ldots,y_N) \in \mathbb{R}^N: \\
                & x \oplus y := (x_1+y_1,x_2+y_2,\ldots,x_N+y_N) \in \mathbb{R}^N
            \end{split}
        \end{equation*}

        \begin{equation*}
            \begin{split}
                \forall  \alpha{} & \in \mathbb{R}, \forall x = ( x_1, \ldots, x_N) \in \mathbb{R}^N: \\
                & \alpha \odot x := (\alpha x_1,\ldots,\alpha x_N) \in \mathbb{R}^N.
            \end{split}
        \end{equation*}
    \end{example}


    \begin{example}
        \hfill \break
        $ K $ - dowolne ciało \\
        $ (K^N = \underbrace{K \times \ldots \times K}_{N}, \mathbb{R}, \oplus, \odot) $ - przestrzeń wektorowa z działaniami:

        \begin{equation*}
            \begin{split}
                \forall  x ={} & ( x_1, \ldots, x_N), y = (y_1, \ldots,y_N) \in K^N: \\
                & x \oplus y := (x_1+y_1,x_2+y_2,\ldots,x_N+y_N)
            \end{split}
        \end{equation*}

        \begin{equation*}
            \begin{split}
                \forall  \alpha{} & \in K, \forall x = ( x_1, \ldots, x_N) \in K^N: \\
                & \alpha \odot x := (\alpha x_1,\ldots,\alpha x_N).
            \end{split}
        \end{equation*}
    \end{example}

    % \begin{example}

    % \end{example}

    % \begin{example}

    % \end{example}

    % \begin{example}

    % \end{example}

    % \begin{lemat}

    % \end{lemat}

    % \begin{lemat}

    % \end{lemat}

    % \begin{lemat}

    % \end{lemat}

    % \begin{lemat}

    % \end{lemat}

    % section przestrzenie_wektorowe (end)

    \section{Podprzestrzenie wektorowe} % (fold)
    \label{sec:podprzestrzenie_wektorowe}

    % \begin{definition}

    % \end{definition}

    % \begin{theorem}

    % \end{theorem}

    % \begin{example}

    % \end{example}

    % \begin{example}

    % \end{example}

    % \begin{example}

    % \end{example}

    % \begin{example}

    % \end{example}

    % \begin{theorem}

    % \end{theorem}

    % \begin{definition}

    % \end{definition}

    % \begin{definition}

    % \end{definition}

    % \begin{example}

    % \end{example}

    % section podprzestrzenie_wektorowe (end)

    \section{Liniowa niezależność wektorów} % (fold)
    \label{sec:liniowa_niezależność_wektorów}

    % section liniowa_niezależność_wektorów (end)

    \section{Baza} % (fold)
    \label{sec:baza}

    % section baza (end)
    \newpage
    \section{Wymiar przestrzeni wektorowej} % (fold)
    \label{sec:wymiar_przestrzeni_wektorowej}

    \begin{theorem}
        Wszystkie bazy danej przestrzeni wektorowej są równoliczne.
    \end{theorem}

    \begin{conclusion}
        ( twierdzenie Steinera ) \\
        Jeśli pewna baza przestrzeni $ V $ ma $ n $ elementów, to każda inna baza tej przestrzeni ma też $ n $ elementów.
    \end{conclusion}

    \begin{definition} \hfill
        \begin{itemize}
            \item Jeśli przestrzeń wektorowa $ V $ ma skończoną bazę, to mówimy, że jest skończenie wymiarowa i oznaczamy $ dim V $.
            \item Jeśli przestrzeń V ma nieskończoną bazę, to mówimy, że jest nieskończenie wymiarowa i wówczas oznaczamy $ dim V = \infty $.
        \end{itemize}
    \end{definition}

    \large{\textbf{Uwaga:}}
    Powyższa definicja jest poprawna ponieważ wszystkie bazy są równoliczne.


    \begin{theorem} \hfill \\\\
        \textbf{Założenia: } $ V $ w przestrzeni wektorowej, $ V_1 \subseteq V $ - podprzestrzeń
        \textbf{Teza: }
        \begin{enumerate}
            \item $ dim V_1 \leq dim V $
            \item Jeśli $ dim V_1 = dim < \infty $, to $ V = V_1 $ ??
        \end{enumerate}
    \end{theorem}

    \large{\textbf{Dowód.}} \\
    \begin{enumerate}
        \item Niech $B$ będzie bazą $ V_1 \implies dim V_1 = \sharp B $ \\
              $ \implies B $ jest zbiorem wektorów liniowo niezależnych w $ V $ \\
              $ \implies B $ można rozszerzyć do bazy \\
              $ A \supseteq B $ w $ V $ \\
              $ dim V = \sharp A \geq \sharp B = dim V_1 $
        \item Niech $B$ będzie bazą w $V_1$ \\
              $ \implies B $ mogę rozszerzyć do bazy $ A \geq B $ w $V$ \\
              ale  $ \sharp A = \sharp B < \infty \implies A = B $ \\

              $ V_1 = lin( B ) = lin( A ) = V  $
    \end{enumerate}

    \begin{flushright}
        $ \Box $
    \end{flushright}


    % section wymiar_przestrzeni_wektorowej (end)

	\section{Odwzorowania liniowe}
    \label{sec:odwzorowania_liniowe}

    \begin{definition} \hfill \\\\
        Niech $ V $ i $ W $ będą przestrzeniami wektorowymi nad ciałem $ K $.
        Mówimy, że odwzorowanie $ \varphi: V \mapsto W $ jest \textbf{liniowe}, jeśli

        \begin{equation*}
            \forall x, y \in V, \alpha, \beta \in K: \varphi(\alpha x + \beta y)
            = \alpha \varphi(x) + \beta \varphi (y)
        \end{equation*}

    \end{definition}

    \begin{theorem}
        $ \varphi : V \mapsto W $ jest odwzorowaniem liniowym $\Longleftrightarrow$

        \begin{numcases}{}
            \forall x,y \in V: \varphi(x + y) = \varphi(x) + \varphi(y) \\
            \forall \alpha \in K, x \in V: \varphi( \alpha x ) = \alpha \varphi (x)
        \end{numcases}
    \end{theorem}

    \large{\textbf{Dowód.}} \\
    "$ \Longleftarrow $" \\
    \begin{equation*}
        \varphi( \alpha x + \beta y ) \one \varphi( \alpha x ) + \varphi( \beta y )
        \two \alpha \varphi(x) + \beta \varphi(y)
    \end{equation*}

    "$ \Longrightarrow $" \\
        \begin{itemize}
            \item
                $ \varphi(x + y) = \varphi (1 \cdot x + 1 \cdot y )
                \ass 1 \cdot \varphi(x) + 1 \cdot \varphi(y)
                = \varphi(x) + \varphi(y) $
            \item
                $ \varphi(\alpha x) = \varphi( \alpha x + 0 \cdot x)
                \ass \alpha \varphi(x) + 0 \cdot \varphi(x)
                = \alpha \varphi(x) + \Theta = \alpha \varphi(x) $
        \end{itemize}
        \begin{flushright}
            $ \Box $
        \end{flushright}

    \begin{theorem}
        $ \varphi: V \mapsto W $ jest liniowe $ \Longleftrightarrow $ \\
        \begin{equation*}
            \forall k \in \mathbb{N}, \forall v_1, \ldots, v_k \in V,
            \forall \alpha_1, \ldots, \alpha_k \in K:
            \varphi(\sum_{i = 1}^{k} \alpha_i v_i ) = \sum_{i = 1}^{k} \alpha_i \varphi(v_1)
        \end{equation*}
    \end{theorem}

    \large{\textbf{Dowód.}} indukcją matematyczną na $k$ (ćw.)

    \begin{theorem} \hfill \\\\
        Jeśli $ \varphi: V \mapsto W $ jest odwzrowaniem liniowym to
        $ \varphi(\Theta_v) = \Theta_w $
    \end{theorem}

    \large{\textbf{Dowód.}} \\
    \begin{equation*}
        \varphi(\Theta_v) = \varphi(0 \cdot v) = 0 \cdot \varphi(v) = \Theta_w
    \end{equation*}

    \begin{flushright}
        $ \Box $
    \end{flushright}

    \begin{example}
    \end{example}

    \begin{enumerate}
        \item
            $ id_v: V \ni v \longmapsto v \in V $ jest odwzorowaniem liniowym
            \begin{equation*}
                id_v(\alpha v_1 + \beta v_2) = \alpha v_1 + \beta v_2
                = \alpha \cdot id_v (v_1) + \beta \cdot id(v_2)
            \end{equation*}
        \item
            \textbf{Odwzorowanie zerowe} \\
            \begin{equation*}
                \varphi: V \ni v \longmapsto \Theta_w \in W
            \end{equation*}
            \begin{equation*}
                \Theta(\alpha v_1 + \beta v_2) = \Theta_w
                = \alpha \cdot \Theta_w + \beta \cdot \Theta_w
                = \alpha \varphi(v_1) + \beta \varphi(v_2)
            \end{equation*}
        \item
            Niech $ V = V_1 \oplus V_2 $ \\
            Wówczas $ \forall ~v \in V ~ \exists! ~ v_1 \in V_1 ~ \exists! ~ v_1 \in V_2:
            v = v_1 + v_2 $ \\

            Definiujemy odwzorowanie $ \pi_1 : V \ni v \longmapsto v_1 \in V_1 $ \\
            \tab - rzutowanie na $ V_1 $ w kierunku $ V_2 $ \\

            Analogicznie $ \pi_2 : V \ni v \longmapsto v_2 \in V_2 $ \\
            \tab - rzutowanie na $ V_2 $ w kierunku $ V_1 $ \\

            \newpage
            Udowodnimy, że $\pi_1$ jest liniowe (dowód dla $\pi_2$ analogiczny).\\\\
            Niech $ x, y \in V, \alpha , \beta \in K $\\
            Wówczas \\
            \begin{equation*}
                x = x_1 + x_2 \\
            \end{equation*}
            \begin{equation*}
                y = y_1 + y_2
            \end{equation*}
                gdzie $ x_1, y_1 \in V_1, x_2, y_2 \in V_2 $

            \begin{equation*}
                \begin{aligned}
                    \pi_1 \overbrace{(\alpha x + \beta y)}^{v} &= \pi_1 ( \alpha(x_1 + x_2) + \beta(y_1 + y_2))\\
                    & = \pi_1( \underbrace{( \alpha x_1 + \beta y_1 )}_{\in V_1}
                        + \underbrace{( \alpha x_2 + \beta y_2 )}_{\in V_2})\\
                    & = \alpha x_1 + \beta y_1 \\
                    & = \alpha \pi_1(x) + \beta \pi_1(y)
                \end{aligned}
            \end{equation*}

            <<jakis rysuneczek tutaj>>
        \item
            Niech $W$ będzie podprzestrzenią wektorową w $V$\\
            \begin{equation*}
                \kappa: W \ni w \longmapsto w \in V - zanurzenie
            \end{equation*}

            Przykład <<rysuneczek dwie osie i R do 2>>

            \begin{equation*}
                \mathbb{R} \ni x \longmapsto (x,0) \in \mathbb{R}^2
            \end{equation*}
            \begin{equation*}
                \mathbb{R} \times \{0\} \ni (x,0) \longmapsto (x,0) \in \mathbb{R}^2
            \end{equation*}
        \item
            $ V $ - przestrzeń wektorowa, $ \lambda \in K $\\\\
            Odwzorowanie $ \varphi_\lambda: v \ni v \longmapsto \lambda v \in V $
            jest liniowe i nazywamy je \textbf{homotetią}.
        \item
            $ I $ - przedział w $ \mathbb{R} $ \\
            \begin{equation*}
                \mathcal{C}^\infty(I) = \{ f: I \longmapsto \mathbb{R}:
                ~f~~jest~~w~~\mathcal{C}^\infty \}
            \end{equation*}

            \begin{equation*}
                F:~~ \mathcal{C}^\infty(I) \ni f \longmapsto f' \in \mathcal{C}^\infty(I)
            \end{equation*}

            \begin{tabbing}
                Ponieważ \= $ (f+g)' = f' + g' $ \\
                         \> $ (\alpha f)' = \alpha \cdot f' $
            \end{tabbing}

            Zatem $F$ jest odwzorowaniem liniowym.
    \end{enumerate}

    % Jeśli $ V = W $, to $ \varphi $ nazywamy \textbf{endomorfizmem}.\\
    % Jeśli $ V = W $, $ \varphi $ jest bijekcją, to $ \varphi $ nazywamy
    % \textbf{automorfizmem}.
    %
    % \newpage
    % \begin{example}
    % \end{example}
    %
    % \begin{enumerate}
    %     \item
    %         $ id_v $ - izomorfizm (automorfizm)
    %     \item
    %         Odwzorowanie zerowe $ \varphi: V \longmapsto W $ \\
    %         \tab jest monoformizmem $ \Longleftrightarrow V = {\Theta} $\\
    %         \tab jest epimorfizmem $ \Longleftrightarrow W = {\Theta} $
    % \end{enumerate}

\end{document}
