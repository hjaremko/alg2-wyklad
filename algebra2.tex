\documentclass[a5paper,8pt]{article}
\usepackage[utf8]{inputenc}
\usepackage[T1]{fontenc}
\usepackage[MeX]{polski}
\usepackage{amsmath}
\usepackage{amsthm}
\usepackage{amsfonts}
\usepackage{geometry}
\newgeometry{tmargin=2cm, bmargin=2cm, lmargin=1.5cm, rmargin=1.5cm}

\title{Algebra liniowa z geometrią 2}
\date{}
\frenchspacing

\newtheorem{example}{Przykład}[section]
\newtheorem{lemat}{Lemat}[section]
\newtheorem{definition}{Definicja}[section]
\newtheorem{theorem}{Twierdzenie}[section]
\newtheorem*{conclusion}{Wniosek}

\begin{document}
    \maketitle
    % \tableofcontents
    % \pagebreak
    % \date{1 marca 2018}

    \section{Terminy kolowium}

    \begin{itemize}
        \item 12.04 - pierwszy semestr
        \item 24.05 - drugi semestr (1-11) Jurlewicz
    \end{itemize}

    \section{Przestrzenie wektorowe} % (fold)
    \label{sec:przestrzenie_wektorowe}

    \large{\textbf{Uwagi:   }}
    $ (V, K, \oplus, \odot) $ - przestrzeń wektorowa

    \begin{enumerate}
        \item Elementy ciala $K$ nazywamy \textbf{skalarami}, a elementy zbioru $V$ \textbf{wektorami}.
        \item Dla uproszczenia zapisu na skalarach i wektorach działania będziemy oznaczali $ +, \cdot $.
        \item Gdy $ K = \mathbb{R} $, to przestrzeń wektorowa $ (V, \mathbb{R}, +, \cdot) $ nazywamy rzeczywistą.
        \item Gdy $ K = \mathbb{C} $, to $ (V, \mathbb{C}, +, \cdot) $ nazywamy zespoloną.
        \item Zamiast pisać $ (V, K, +, \cdot) $ często piszemy "$V$ jest przestrzenią wektorową".
        \item Jeśli $ x, y \in V $, to zapis $x-y$ oznaczamy $x + (-y)$, gdzie $-y$ jest elementem przeciwnym w grupie addytywnej $(V, +)$.
        \item Element neutralny w grupie $(V, +)$ oznaczamy $\Theta$ i nazywamy wektorem zerowym.
    \end{enumerate}

    \begin{example}
        \hfill \break 
        $ (\mathbb{R}^N = \underbrace{\mathbb{R} \times \ldots \times \mathbb{R}}_{N}, \mathbb{R}, \oplus, \odot) $ jest przestrzenią wektorową, gdzie działania są zdefiniowane następująco:

        \begin{equation*}
            \begin{split}
                \forall  x ={} & ( x_1, \ldots, x_N), y = (y_1, \ldots,y_N) \in \mathbb{R}^N: \\
                & x \oplus y := (x_1+y_1,x_2+y_2,\ldots,x_N+y_N) \in \mathbb{R}^N
            \end{split}
        \end{equation*}

        \begin{equation*}
            \begin{split}
                \forall  \alpha{} & \in \mathbb{R}, \forall x = ( x_1, \ldots, x_N) \in \mathbb{R}^N: \\
                & \alpha \odot x := (\alpha x_1,\ldots,\alpha x_N) \in \mathbb{R}^N.
            \end{split}
        \end{equation*}
    \end{example}


    \begin{example}
        \hfill \break 
        $ K $ - dowolne ciało \\
        $ (K^N = \underbrace{K \times \ldots \times K}_{N}, \mathbb{R}, \oplus, \odot) $ - przestrzeń wektorowa z działaniami:

        \begin{equation*}
            \begin{split}
                \forall  x ={} & ( x_1, \ldots, x_N), y = (y_1, \ldots,y_N) \in K^N: \\
                & x \oplus y := (x_1+y_1,x_2+y_2,\ldots,x_N+y_N)
            \end{split}
        \end{equation*}

        \begin{equation*}
            \begin{split}
                \forall  \alpha{} & \in K, \forall x = ( x_1, \ldots, x_N) \in K^N: \\
                & \alpha \odot x := (\alpha x_1,\ldots,\alpha x_N).
            \end{split}
        \end{equation*}
    \end{example}

    % \begin{example}

    % \end{example}

    % \begin{example}

    % \end{example}

    % \begin{example}

    % \end{example}
    
    % \begin{lemat}

    % \end{lemat}

    % \begin{lemat}

    % \end{lemat}

    % \begin{lemat}

    % \end{lemat}

    % \begin{lemat}

    % \end{lemat}

    % section przestrzenie_wektorowe (end)

    \section{Podprzestrzenie wektorowe} % (fold)
    \label{sec:podprzestrzenie_wektorowe}

    % \begin{definition}

    % \end{definition}

    % \begin{theorem}

    % \end{theorem}

    % \begin{example}

    % \end{example}

    % \begin{example}

    % \end{example}

    % \begin{example}

    % \end{example}

    % \begin{example}

    % \end{example}

    % \begin{theorem}

    % \end{theorem}

    % \begin{definition}

    % \end{definition}

    % \begin{definition}

    % \end{definition}

    % \begin{example}

    % \end{example}
    
    % section podprzestrzenie_wektorowe (end)

    \section{Liniowa niezależność wektorów} % (fold)
    \label{sec:liniowa_niezależność_wektorów}
    
    % section liniowa_niezależność_wektorów (end)

    \section{Baza} % (fold)
    \label{sec:baza}
    
    % section baza (end)

    \section{Wymiar przestrzeni wektorowej} % (fold)
    \label{sec:wymiar_przestrzeni_wektorowej}
    
    \begin{theorem}
        Wszystkie bazy danej przestrzeni wektorowej są równoliczne.
    \end{theorem}

    \begin{conclusion}
        ( twierdzenie Steinera ) \\
        Jeśli pewna baza przestrzeni $ V $ ma $ n $ elementów, to każda inna baza tej przestrzeni ma też $ n $ elementów.
    \end{conclusion}

    \begin{definition} \hfill
        \begin{itemize}
            \item Jeśli przestrzeń wektorowa $ V $ ma skończoną bazę, to mówimy, że jest skończenie wymiarowa i oznaczamy $ dim V $.
            \item Jeśli przestrzeń V ma nieskończoną bazę, to mówimy, że jest nieskończenie wymiarowa i wówczas oznaczamy $ dim V = \infty $.
        \end{itemize}
    \end{definition}

    \large{\textbf{Uwaga:}}
    Powyższa definicja jest poprawna ponieważ wszystkie bazy są równoliczne.


    \begin{theorem} \hfill \\\\
        \textbf{Założenia: } $ V $ w przestrzeni wektorowej, $ V_1 \subseteq V $ - podprzestrzeń
        \textbf{Teza: }
        \begin{enumerate}
            \item $ dim V_1 \leq dim V $
            \item Jeśli $ dim V_1 = dim < \infty $, to $ V = V_1 $ ??
        \end{enumerate}
    \end{theorem}

    \large{\textbf{Dowód:}} \\
    \begin{enumerate}
        \item Niech $B$ będzie bazą $ V_1 \implies dim V_1 = \#B $ \\
              $ \implies B $ jest zbiorem wektorów liniowo niezależnych w $ V $ \\
              $ \implies B $ można rozszerzyć do bazy \\
              $ A \supseteq B $ w $ V $ \\
              $ dim V = \#A \geq \#B = dim V_1 $
        \item Niech $B$ będzie bazą w $V_1$ \\
              $ \implies B $ mogę rozszerzyć do bazy $ A \geq B $ w $V$ \\
              ale  $ \#A = \#B < \infty \implies A = B $ \\

              $ V_1 = lin( B ) = lin( A ) = V \\ \Box $
    \end{enumerate}

    % section wymiar_przestrzeni_wektorowej (end)

\end{document}